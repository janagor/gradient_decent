
    \section{Część teoretyczna}
    \subsection{Dyskretny problem plecakowy}
    Mamy do dyspozycji plecak o maksymalnej pojemności $W$ oraz zbiór $N$ 
    elementów $\{x_1, ..., x_i, .., x_N\}$, przy czym każdy element ma określoną 
    wartość $p_i$ oraz wage $w_i$. 

    Celem jest wybranie przedmiotów o największej łącznej wartości i łącznej 
    wadze nie większej niż $W$:
    \[
    max \sum_{i=1}^{n} x_i \cdot p_i 
    \]
    \[
      \sum_{i=1}^{n} x_i \cdot p_i \leq W
    \]
    \[
      x_i \in {0,1}
    \]
    Problem plecakowy jest problemem NP trudnym, jego rozwiązanie wielomianowe
    nie jest znane. 

    \subsubsection{Rozwiązanie wyczerpujące}
    Pierwszy algorytm rozwiązujący problem plecakowy polega na przeglądzie 
    wyczerpującym. Analizowane są w nim wszystkie kombinacje (tudzież
    wszystkie podzbiory zbioru $N$) i wybierane jest rozwiązanie optymalne.
    \subsubsection{Rozwiązanie przy użyciu heurystyki}
    Drugi algorytm rozwiązujący problem korzysta z nastepującej heurystyki:
    do plecaka pakowane są przedmioty według kolejności wynikającej ze stosunku
    ich wartości do masy.

    \subsection{Metoda najszybszego wzrostu}
    Niech $\{\beta_t, t = 1,2,3,...\}$ będzie ciągiem liczb dodatnich, $\mathcal{X} = \mathcal{R}^d$, $J: X \rightarrow \mathbb{R}$, zaś $\{x_t, t = 1,2,3,...\}$ - ciągiem elementów $\mathcal{X}$ wyznaczanych w następującej iteracji:
    \[
      x_{t+1} = x_t - \beta_t \cdot \nabla J(x_t), \quad t = 1,2,...
    \]
    dla pewnego $x_1 \in \mathcal{X}$. Procedura ta modyfikuje wektory $x_t$ przeciwnie 
    do kierunku wzrostu funkcji $J$.


